\documentclass{beamer}
\usepackage[utf8]{inputenc}
\usepackage[T1]{fontenc}
\usepackage[francais]{babel}
\usetheme{Singapore} %Boadilla | Bergen | Madrid | Antibes | Hannover | Singapore | Warsaw

%----------------------------------------------------------------------------------------
%   TITLE INFORMATION
%----------------------------------------------------------------------------------------
\title{LaRuche}
\subtitle{HLSE602 -- Projet CMI Annuel}
\author{B. Rima \and O. Farajallah \and W. Soussi}
\institute[UM]{L3 CMI Informatique}
\date{\today}

\begin{document}
%----------------------------------------------------------------------------------------
%   TITLE FRAME
%----------------------------------------------------------------------------------------
\begin{frame}
\titlepage
\end{frame}
%----------------------------------------------------------------------------------------
%   OUTLINE
%----------------------------------------------------------------------------------------
\begin{frame}{Sommaire}
\tableofcontents
\end{frame}
%----------------------------------------------------------------------------------------
%   INTRODUCTION
%----------------------------------------------------------------------------------------
\section{Introduction}
\subsection{Contexte du projet}
\begin{frame}{Contexte du projet}{Introduction}
%Contenu
\end{frame}

%----------------------------------------------------------------------------------------
%   PROBLÉMATIQUE ET MÉTHODOLOGIE DE RÉSOLUTION
%----------------------------------------------------------------------------------------
\section{Problématique et Méthodologie de Résolution}
\subsection{Problématique}
\begin{frame}{Problématique}{Problématique et Méthodologie de Résolution}
%Contenu
\end{frame}

\subsection{Solution proposée : \texttt{LaRuche}}
\begin{frame}{Solution proposée : \texttt{LaRuche}}{Problématique et Méthodologie de Résolution}
%Contenu
\end{frame}

\subsection{Méthodes agiles}
\begin{frame}{Méthodes agiles}{Problématique et Méthodologie de Résolution}
%%User Stories
%%Entretiens
\end{frame}
%----------------------------------------------------------------------------------------
%   CONCEPTION
%----------------------------------------------------------------------------------------
\section{Conception}
\subsection{Côté fournisseur}
\begin{frame}{Côté fournisseur}{Conception}
%%SD : Ruche, Cellule, Voisins, Voisinage imposé, ...
%%Use Cases
%%Storyboard
%%inscription, ruches, et points de collecte
%%définition de produits, gestion des stocks
%%paiement (espèce puis en ligne)
\end{frame}

\subsection{Côté client}
\begin{frame}{Côté client}{Conception}
%%Complexité cachée des ruches du point de vue des clients
%%Use Cases
%%Storyboard
%%inscription, recherche
%%chat
%%réservation de produits, choix de lieu et date de collecte
\end{frame}

\section{Outils d'implémentation}
\subsection{\protect\textit{Front-end}}
\begin{frame}{\textit{Front-end}}{Outils d'implémentation}
%%Bootstrap, JS, JQuery, React
\end{frame}

\subsection{\protect\textit{Back-end}}
\begin{frame}{\textit{Back-end}}{Outils d'implémentation}
%%PHP, MySQL
\end{frame}
%----------------------------------------------------------------------------------------
%   CONCLUSION
%----------------------------------------------------------------------------------------
\section{Conclusion}
\subsection{Écosystème décentralisé et autonome}
\begin{frame}{Écosystème décentralisé et autonome}{Conclusion}
%pas de hiérarchie
%ruches et évènements de collecte déterminées par les fournisseurs
%système versatile et extensible
\end{frame}

\subsection{Perspectives}
\begin{frame}{Perspectives}{Conclusion}
%fonctionnalités déjà discutées : paiement en ligne (CB, Paypal), politique de quasi rupture de stock et autres approches d'optimisation, etc.. (cf. section logistiques et user stories supplémentaires)
%fonctionnalités pondérées non discutées : internationalisation du site, etc...
\end{frame}
\end{document}
