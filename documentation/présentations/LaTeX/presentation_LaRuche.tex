\documentclass{beamer}
\usepackage[utf8]{inputenc}
\usepackage[T1]{fontenc}
\usepackage[francais]{babel}
%\usepackage{enumitem}
\usetheme{Singapore} %Boadilla | Bergen | Madrid | Antibes | Hannover | Singapore | Warsaw

%----------------------------------------------------------------------------------------
%   TITLE INFORMATION
%----------------------------------------------------------------------------------------
\title{LaRuche}
\subtitle{HLSE602 -- Projet CMI Annuel}
\author{B. Rima \and O. Farajallah \and W. Soussi}
\institute[UM]{L3 CMI Informatique}
\date{\today}

\begin{document}
%----------------------------------------------------------------------------------------
%   TITLE FRAME
%----------------------------------------------------------------------------------------
\begin{frame}
\titlepage
\end{frame}
%----------------------------------------------------------------------------------------
%   OUTLINE
%----------------------------------------------------------------------------------------
\begin{frame}{Sommaire}
\tableofcontents
\end{frame}
%----------------------------------------------------------------------------------------
%   INTRODUCTION
%----------------------------------------------------------------------------------------
\section{Introduction}
\begin{frame}{Contexte du projet}{Introduction}
  \begin{description}
    \item[Projet CMI:] Projet annuel pour l'année 2017/2018 sous le parcours du CMI informatique.
    \item[Tuteur:] Le tuteur du projet est l'enseignant chercheur Eric Bourreau.
    \item[Lieux de travail :] La conception et le développement du projet ce fait entre la FDS et le LIRMM.
  \end{description}
\end{frame}

%----------------------------------------------------------------------------------------
%   PROBLÉMATIQUE ET MÉTHODOLOGIE DE RÉSOLUTION
%----------------------------------------------------------------------------------------
\section{Problématique et Méthodologie de Résolution}
\subsection{Problématique}
\begin{frame}{Problématique}{Problématique et Méthodologie de Résolution}
  \begin{description}
    \item[Consommateurs :] acheter des produits frais et minimiser les étapes de processing.
    \item[Producteurs :] se libérer des centres d'achat et des intermédiaires de distribution.
  \end{description}
\end{frame}

\subsection{Solutions}
\begin{frame}{Solution possible : La Ruche Qui Dit Oui}{Problématique et Méthodologie de Résolution}
\begin{block}{Site Web}
Une interface directe entre \textbf{consommateurs} et \textbf{fournisseurs}.
\end{block}

\begin{definition}[Ruche]
Un regroupement de plusieurs membres \textbf{consommateurs} et \textbf{fournisseurs} d'une région, guidé par un \textbf{responsable de ruche}.
\end{definition}

\begin{block}{Vision Centralisée}
\begin{itemize}
  \item l'ensemble des ruches obeit à une \textbf{Ruche-Mama}.
  \item les besoins de chaque ruche sont transmis à la \textbf{Ruche-Mama} via le responsable de ruche correspondant.
  \item la \textbf{Ruche-Mama} s'occupe de la gestion des ruches : création, réglementations internes, interactions, évolution et extensibilité des services, $\dots$.
\end{itemize}
\end{block}
\end{frame}

\begin{frame}{Solution proposée : \texttt{LaRuche}}{Problématique et Méthodologie de Résolution}
\begin{block}{Site Web}
Une interface directe entre \textbf{consommateurs} et \textbf{fournisseurs}.
\end{block}

\begin{definition}[Ruche]
Un regroupement de plusieurs \textbf{fournisseurs} d'une région, \textbf{sans guide explicite} préfixé par le site.
\end{definition}

\begin{block}{Vision Décentralisée et Autonome}
\begin{itemize}
  \item l'ensemble des ruches ne répond à \textbf{aucune entité centralisée fédéral}.
  \item chaque ruche s'occupe de ses propres besoins et de leur gestion sans besoin d'un intermédiaire et d'une hiérarchie autoritaire à respecter.
\end{itemize}
\end{block}
\end{frame}

\subsection{Méthodes agiles}
\begin{frame}{Méthodologie de résolution : méthodes agiles}{Problématique et Méthodologie de Résolution}
\begin{block}{Méthodes Agiles}
Une approche de développement logiciel de plus en plus prépondérante basée sur une conception/développement itérative, orientée-test et orientée-client.
\end{block}

\begin{block}{Pourquoi ?}
\begin{itemize}
  \item meilleure gestion des ressources
  \item sortie plus fréquente de versions fonctionnelles et testées du produit
  \item interaction plus fréquente avec les clients : adaptation et extensibilité du produit selon leurs besoins
\end{itemize}
\end{block}
\end{frame}

\begin{frame}{Méthodologie de résolution : méthodes agiles}{Problématique et Méthodologie de Résolution}
\begin{block}{\textit{User Stories}}
Des requis fournis par les clients, décrivant en langage naturel les fonctionnalités qu'ils souhaitent avoir dans le produit développé.
\end{block}
\end{frame}
%----------------------------------------------------------------------------------------
%   CONCEPTION
%----------------------------------------------------------------------------------------
\section{Conception}
\subsection{Outils de conception utilisés}
\begin{frame}{Outils de conception utilisés}{Conception}
\begin{block}{Diagrammes de cas d'usage}
Des diagrammes dynamiques, souvent utilisés en \textbf{UML} pour décrire en haut niveau des fonctionnalités d'un système.
\end{block}

\begin{block}{Modèle EA}
Un modèle \textbf{conceptuel} utilisé pour décrire les entités du projet et les associations décrivant leurs relations et comportements.
\end{block}

\begin{block}{Schéma de base de données}
Un schéma en modèle \textbf{relationnel} traduit à partir du \textbf{modèle EA} et servant comme \textbf{support} lors de l'implémentation de la \textbf{base de données}.
\end{block}

\begin{block}{\textit{mockup storyboard}}
Un document de haut niveau (peu de détails sur les fonctionnalités) pour schématiser l'utilisation d'un projet.
\end{block}
\end{frame}

\subsection{Côté fournisseur}
\begin{frame}{Côté fournisseur}{Conception}
%%SD : Ruche, Cellule, Voisins, Voisinage imposé, ...
%%Use Cases
%%Storyboard
%%inscription, ruches, et points de collecte
%%définition de produits, gestion des stocks
%%paiement (espèce puis en ligne)
\end{frame}

\subsection{Côté client}
\begin{frame}{Côté client}{Conception}
%%Complexité cachée des ruches du point de vue des clients
%%Use Cases
%%Storyboard
%%inscription, recherche
%%chat
%%réservation de produits, choix de lieu et date de collecte
\end{frame}

\section{Outils d'implémentation}
\subsection{\protect\textit{Front-end}}
\begin{frame}{\textit{Front-end}}{Outils d'implémentation}
%%Bootstrap, JS, JQuery, React
\end{frame}

\subsection{\protect\textit{Back-end}}
\begin{frame}{\textit{Back-end}}{Outils d'implémentation}
%%PHP, MySQL
  \begin{description}
    \item[Language de développement :] Serveur codé en PHP et le framework Symfony pour un système modulaire et basé sur le design pattern MVC.
    \item[Base de données :] Base de données relationnel mis en place avec MySQL. Interaction avec l'ORM Doctrine.
  \end{description}
\end{frame}
%----------------------------------------------------------------------------------------
%   CONCLUSION
%----------------------------------------------------------------------------------------
\section{Conclusion}
\subsection{Écosystème décentralisé et autonome}
\begin{frame}{Écosystème décentralisé et autonome}{Conclusion}

\begin{block}{Autonomie}
Les fournisseurs sont \textbf{autonomes} et ils déterminent la formation des ruches et des évènements de collecte.
\end{block} 
\begin{block}{Extensibilité}
Le système est versatile et extensible, adapte pour l'extension des ruches ou l'augmentation de nombre de ces derniers.
\end{block} 

\end{frame}

\subsection{Perspectives}
\begin{frame}{Perspectives}{Conclusion}
%fonctionnalités déjà discutées : paiement en ligne (CB, Paypal), politique de quasi rupture de stock et autres approches d'optimisation, etc.. (cf. section logistiques et user stories supplémentaires)
%fonctionnalités pondérées non discutées : internationalisation du site, etc...
\end{frame}
\end{document}
