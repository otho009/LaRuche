\documentclass[usenames,dvipsnames]{beamer}
\usepackage[utf8]{inputenc}
\usepackage[T1]{fontenc}
\usepackage[french]{babel}
\usepackage{xcolor}
\usetheme{Singapore} %Boadilla | Bergen | Madrid | Antibes | Hannover | Singapore | Warsaw
%----------------------------------------------------------------------------------------
%   TITLE INFORMATION
%----------------------------------------------------------------------------------------
\title{CourtCircuit}
\subtitle{HLSE602 -- Projet CMI Annuel}
\author{B. Rima \and O. Farajallah \and W. Soussi}
\institute[UM]{L3 CMI Informatique}
\date{\today}

\begin{document}
%----------------------------------------------------------------------------------------
%   TITLE FRAME
%----------------------------------------------------------------------------------------
\begin{frame}
\titlepage
\end{frame}
%----------------------------------------------------------------------------------------
%   OUTLINE
%----------------------------------------------------------------------------------------
\begin{frame}{Sommaire}
\tableofcontents
\end{frame}
%----------------------------------------------------------------------------------------
%   INTRODUCTION
%----------------------------------------------------------------------------------------
\section{Introduction}
\begin{frame}{Contexte du projet}{Introduction}
  \begin{description}
    \item [Projet CMI :] Module d'un projet annuel pour l'année 2017--2018 dans le cadre du \textbf{CMI Informatique}
    \item [Responsable CMI Informatique :] Mme Anne-Elisabeth Baert
    \item [Encadrant du projet :] M. Eric Bourreau
    \item [Lieux de travail :] La \textbf{FDS} et le \textbf{LIRMM}
  \end{description}
\end{frame}

%----------------------------------------------------------------------------------------
%   RAPPELS
%----------------------------------------------------------------------------------------
\section{Rappels}
\subsection{Problématique}
\begin{frame}{Rappels}{Problématique}
\begin{columns}[onlytextwidth, T]
  \column{50mm}
    \includegraphics[scale=0.19]{images/chain_production.png}

  \column{\dimexpr\linewidth-35mm-4mm}
    \begin{block}{Consommateurs :}
    Acheter des produits frais et minimiser \\les étapes de processing.
    \end{block}

    \begin{block}{Producteurs :}
    Maîtriser le prix de vente et \\les débouchés de leurs productions en \\se libérant des intermédiaires de distribution.
    \end{block}
\end{columns}
\end{frame}

\subsection{Solution proposée : \texttt{CourtCircuit}}
\begin{frame}{Rappels}{Solution proposée : \texttt{CourtCircuit}}
  \begin{block}{Site web \textit{e-commerce}}
  Une interface directe entre \textbf{consommateurs} et \textbf{fournisseurs}.
  \end{block}

  \begin{block}{Ruche}
  Un regroupement de plusieurs \textbf{fournisseurs} d'une région, \textbf{sans guide explicite} préfixé par le site, associé à plusieurs points de collecte.
  \end{block}

  \begin{block}{Vision Décentralisée et Autonome}
  \begin{itemize}
    \item l'ensemble des ruches ne répond à \textbf{aucune entité centrale}.
    \item chaque ruche s'occupe de ses propres besoins et de leur gestion sans besoin d'un intermédiaire et d'une hiérarchie à respecter.
  \end{itemize}
  \end{block}
\end{frame}

\subsection{Outils de conception}
\begin{frame}{Rappels}{Outils de conception}
  \begin{enumerate}
    \item \textit{User Stories} (outil de conception agile)
    \item Diagrammes de cas d'usage
    \item Modèle EA
    \item Schéma de base de données
    \item \textit{Storyboard}
  \end{enumerate}
\end{frame}
%----------------------------------------------------------------------------------------
%   IMPLÉMENTATION
%----------------------------------------------------------------------------------------
\section{Implémentation}
\subsection{Outils d'implémentation}
\begin{frame}{Outils d'implémentation}{Implémentation}
  \begin{figure}[!ht]
    \centering
    \includegraphics[scale=0.3]{images/website_architecture.jpeg}
  \end{figure}

  \begin{description}
    \item [\textit{Front-end} :] React.js, JSX, Bootstrap
    \item [\textit{Back-end} :] Node.js, Express.js
    \item [Base de données :] MariaDB
  \end{description}
\end{frame}

\subsection{Application web monopage (SPA)}
\begin{frame}{Application web monopage (SPA)}{Implémentation}
schéma d'une SPA
\end{frame}

\begin{frame}{Application web monopage (SPA)}{Implémentation}
pros et cons d'une SPA
\end{frame}

\subsection{\protect\textit{Front-end}}
\begin{frame}{\textit{React}}{Front-end}

\end{frame}

\begin{frame}{\textit{Bootstrap et Font-Awesome}}{Front-end}

\end{frame}

\begin{frame}{\textit{Charte Graphique}}{Front-end}

\end{frame}

\subsection{\protect\textit{Back-end}}
\begin{frame}{\textit{Node.js} (Introduction)}{Back-end}
  \begin{figure}[!ht]
    \includegraphics[scale=0.03]{images/nodejs_icon.png}
  \end{figure}

  \begin{itemize}
    \item environnement d'exécution \textbf{JavaScript} côté serveur utilisant le moteur \textbf{JavaScript V8} de \textit{Google Chrome}.
    \item gratuit et \textit{open-source}.
    \item modélisation événementielle, monothread et non-bloquante.
    \item architecture modulaire.
    \item gestionnaire de paquets \textbf{NPM} (\textit{Node Package Manager}) $\rightarrow$ facilité d'usage et d'extensibilité.
  \end{itemize}
\end{frame}

\begin{frame}{\textit{Node.js} (Raisons du choix)}{Back-end}
  \begin{itemize}
    \item écrire du code \textbf{JavaScript} du côté serveur $\rightarrow$ un seul langage pour les côtés client et serveur.
    \item modélisation événementielle, monothread et non-bloquante $\rightarrow$ performance fluide et gestion efficace d'un ensemble important de données.
    \item ensemble important de modules utilitaires facilement téléchargeable via \textbf{NPM}.
  \end{itemize}
\end{frame}

\begin{frame}{\textit{Node.js} (Utilisation)}{Back-end}
  \begin{itemize}
    \item création d'une \textbf{API} factorisée, non redondante et facilement lisible (\textit{Client, Utilisateur, Produit, $\dots$}) permettant d'interfacer avec la base de données.
    \item héberger \textbf{Express}.
  \end{itemize}
\end{frame}

\begin{frame}{\textit{Express} (Introduction)}{Back-end}
  \begin{figure}[!ht]
    \centering
    \includegraphics[scale=0.25]{images/express_icon.png}
  \end{figure}

  \begin{itemize}
    \item framework web minimaliste pour \textbf{Node.js}.
    \item gratuit et \textit{open-source}.
    \item utilisation de \textit{middleware}.
    \item gestion des routes \textbf{REST} (\textit{Representational State Transfer}) et des formulaires en s'appuyant sur des concepts du modèle \textbf{MVC}.
    \item moteurs de templates (\textit{EJS (Embedded JavaScript), Pug, Handlebars, $\dots$}).
  \end{itemize}
\end{frame}

\begin{frame}{\textit{Express} (Raisons du choix et utilisation)}{Back-end}
  \begin{itemize}
    \item framework web \textit{de-facto} pour \textbf{Node.js}.
    \item réduire la verbosité du code \textbf{Node.js} natif pour la création du serveur \textbf{HTTP}.
    \item utilisation de \textit{middleware} pour le traitement des requêtes clients.
    \item gestion des routes \textbf{REST} pour les opérations \textbf{CRUD}.
  \end{itemize}
\end{frame}

\begin{frame}{\textit{MariaDB}}{Back-end}

\end{frame}
%----------------------------------------------------------------------------------------
%   RÉSULTATS ET DIFFICULTÉS SURVENUES
%----------------------------------------------------------------------------------------
\section{Résultats}
\subsection{Ce qui marche et ce qui ne marche pas}
\begin{frame}{Ce qui marche et ce qui ne marche pas}{Résultats}

\end{frame}

\subsection{Difficultés survenues}
\begin{frame}{Difficultés survenues}{Résultats}
  \begin{itemize}
    \item Nouveaux concepts et outils d'implémentation nécessitant un temps d'apprentissage considérable
    \item Temps dédié à l'implémentation insuffisant
    \item Problèmes liés au serveur d'hébergement
  \end{itemize}
\end{frame}
%----------------------------------------------------------------------------------------
%   CONCLUSION
%----------------------------------------------------------------------------------------
\section{Conclusion}
\subsection{Apports personnels du projet}
\begin{frame}{Apports personnels du projet}{Conclusion}
  \begin{itemize}
    \item Apports personnels = difficultés survenues
    \item Apprentissage d'outils d'implémentation récents et en pleine évolution
    \item Appréciation plus profonde du langage JavaScript
  \end{itemize}
\end{frame}

\subsection{Perspectives}
\begin{frame}{Perspectives}{Conclusion}
  \begin{itemize}
    \item Continuation du projet au niveau personnel
    \item Récolte de \textit{feedback} des utilisateurs potentiels
    \item Optimisation de la logistique
    \item Implémentation de fonctionnalités supplémentaires (paiement en ligne, commandes retardées, portefeuille virtuel, $\dots$)
    \item Internationalisation
 \end{itemize}
\end{frame}
\end{document}
