\documentclass[a4paper,12pt]{book}
\usepackage[utf8]{inputenc}
\usepackage[francais]{babel}
\usepackage[T1]{fontenc}
\usepackage{graphicx}
\usepackage[colorlinks,urlcolor=blue]{hyperref} %hyperlinks
\usepackage{xcolor} %color text
\usepackage[amsthm]{ntheorem} %theorems and co.
\usepackage{amsmath} %mathematical symbols
\usepackage{amssymb} %squares in itemize
\usepackage{enumitem} %change numbers into letters in enumerate
\usepackage{footmisc} %reference footnotes
\usepackage{natbib} %bibligraphy
\usepackage[nottoc,notlot,notlof]{tocbibind} %bind the table of contents to the bibligoraphy
\usepackage{packages/tikz-uml} %UML elements
\usepackage{packages/tikz-er2} %ER Model elements
\usetikzlibrary{positioning}
\usetikzlibrary{arrows}
\usetikzlibrary{shadows}

%Configuring shapes in the ER Model
\tikzstyle{every entity} = [top color=white, bottom color=blue!30, draw=blue!50!black!100, drop shadow]
\tikzstyle{every attribute} = [top color=white, bottom color=yellow!20, draw=yellow, node distance=7em, drop shadow]
\tikzstyle{every relationship} = [top color=white, bottom color=red!20, draw=red!50!black!100, drop shadow]
\tikzstyle{every edge} = [link]

%Defining custom environments
\theoremstyle{break}
\newtheorem{userStory}{User Story}

\theoremstyle{break}
\newtheorem{constraint}{Règle}

\theoremstyle{definition}
\newtheorem{example}{Exemple}

\theoremstyle{remark}
\newtheorem{remark}{\textbf{Remarque}}

%----------------------------------------------------------------------------------------
%   TITLE PAGE
%----------------------------------------------------------------------------------------
\title{\textbf{HLSE602 -- Projet Annuel CMI} : \textbf{LaRuche} (Compte Rendu \# $1$)}
\author{Bachar Rima, Othmane Farajallah, Wissem Soussi}
\date{\today}

\begin{document}
\pagestyle{plain}

\maketitle

{
  \hypersetup{linkcolor=black}
  \tableofcontents
}

%----------------------------------------------------------------------------------------
%   INTRODUCTION
%----------------------------------------------------------------------------------------
\chapter{Introduction}
%----------------------------------------------------------------------------------------
%   CONTEXTE DU STAGE
%----------------------------------------------------------------------------------------
\section{Contexte du projet}
Dans ce compte rendu, nous nous consacrons à la description détaillée de la phase de conception du projet intitulé \texttt{LaRuche} à effectuer au sein du \textbf{LIRMM} (\textbf{L}aboratoire d'\textbf{I}nformatique, de \textbf{R}obotique et de \textbf{M}icroélectronique de \textbf{M}ontpellier) dans le cadre du module \textbf{HLSE602 -- Projet Annuel CMI} de la $3$\ieme{} année de licence en \textbf{CMI} (\textbf{C}ursus \textbf{M}aster \textbf{I}ngénierie).\\
Le projet se déroule sous l'encadrement de Mme Anne-Elisabeth Baert en tant que responsable de la formation CMI informatique et M. Eric Bourreau, enseignant/chercheur au sein du LIRMM dans l'équipe \textbf{MAORE} (\textbf{M}éthodes \textbf{A}lgorithmes pour l'\textbf{O}rdonnancement et les \textbf{Ré}seaux) \citep{EricBourreauPres}, en tant que responsable pédagogique et encadrant du projet.

Le sujet du projet couvre la création d'un site web dédiée comme interface de communication entre fournisseurs de produits locaux et leurs clients. Il est inspiré du site \href{https://laruchequiditoui.fr/fr}{La Ruche Qui Dit Oui} traitant le même thème et répondant aux mêmes besoins, mais cherche à faire les choses d'une façon différente, surtout au niveau de la logistique et l'architecture du site, afin de fournir une vision différente, voire plus optimisée de la gestion des interactions directes entre clients et fournisseurs.

Nous avons envisagé, lors du déroulement du projet, expérimenter en génie logiciel, afin de prendre ultérieurement une décision éclairée sur la branche de spécialité que je souhaite cibler à l'avenir. Par ailleurs, j’ai considéré que çà serait une bonne occasion pour acquérir de nouvelles savoir-faire et savoir-être pour rajouter plus de valeur à mon CV, mais aussi pour collaborer au développement d'un projet mené par un établissement de recherche et surtout pour créer un réseau significatif de connaissances au sein du LIRMM.

Nous commencerons ce compte rendu en annonçant le contexte du projet, puis nous présenterons LIRMM, les problématiques adressées et traitées dans le cadre du projet, la méthodologie adoptée pour modéliser le problème et y proposer des solutions ainsi que les outils de modélisation et le planning prévisionnel pour répartir les tâches à effectuer dans un cadre spatio-temporel valable. Enfin, nous conclurons en discutant l'implémentation prévue de l'application modélisée et les perspectives, et en listant un bilan.

%----------------------------------------------------------------------------------------
%   PRÉSENTATION DU LIRMM
%----------------------------------------------------------------------------------------
\section{Présentation du LIRMM}
\og Le [...] – LIRMM – est une unité mixte de recherche, dépendant conjointement de l'Université Montpellier et du Centre National de la Recherche Scientifique [(CNRS)]. Il est situé sur le Campus Saint-Priest de l'UM [(Figure \ref{fig:lirmmPhoto})].

\begin{figure}[!ht]
  \centering
  \includegraphics[scale=0.9]{images/lirmmPhoto.jpg}
  \caption{bâtiment 3 du LIRMM, Campus St. Priest}
  \label{fig:lirmmPhoto}
\end{figure}

Les travaux sont menés dans trois départements scientifiques de recherche, [(L’Informatique, La Robotique, et La Microélectronique)] eux-mêmes organisés en \og équipes-projet \fg{}.

Les recherches menées au LIRMM trouvent généralement une finalisation dans des domaines applicatifs aussi divers que la biologie, la chimie, les télécommunications, la santé, l'environnement... et dans les domaines propres du laboratoire : l'informatique, l'électronique et l'automatique.

Ses activités de recherche [le] positionnent [...] pleinement au coeur des sciences et technologies de l’information, de la communication et des systèmes. [En particulier,] les thématiques du département Informatique s’étendent des frontières des mathématiques à la recherche appliquée : algorithmique des graphes, bioinformatique, cryptographie, réseaux, bases de données et systèmes d'information [...], génie logiciel [...], intelligence artificielle [...], interaction homme-machine [...]. \fg{} \citep{lirmmPres}

%----------------------------------------------------------------------------------------
%   PROBLÈME, MÉTHODOLOGIE, OUTILS ET PLANNING
%----------------------------------------------------------------------------------------
\chapter{Problème, Méthodologie, Outils et Planning}
%----------------------------------------------------------------------------------------
%   PROBLÈME
%----------------------------------------------------------------------------------------
\section{Problème}

%----------------------------------------------------------------------------------------
%   MÉTHODOLOGIE
%----------------------------------------------------------------------------------------
\section{Méthodologie}

%----------------------------------------------------------------------------------------
%   OUTILS
%----------------------------------------------------------------------------------------
\section{Outils}

%----------------------------------------------------------------------------------------
%   PLANNING PRÉVISIONNEL
%----------------------------------------------------------------------------------------
\section{Planning prévisionnel}
  %OTHMANE

%----------------------------------------------------------------------------------------
%   CONCEPTION
%----------------------------------------------------------------------------------------
\chapter{Conception}
%----------------------------------------------------------------------------------------
%   USER STORIES
%----------------------------------------------------------------------------------------
\section{\textit{User Stories}}
Dans cette section nous illustrons les \textit{user stories} que nous avons rédigés pour identifier les fonctionnalités du système conçu :
\subsection{Index Page}
\begin{userStory}[Account Creation]
\textbf{As a {\color{green} client}/{\color{red} vendor} user}, I want to have my own \textbf{personal account},\\
\indent
\textbf{so that} I can have my own \textbf{preferences} and my \textbf{history of purchases/sales}.
\end{userStory}

\begin{userStory}[Account Creation via Other Platforms]
\textbf{As a {\color{green} client}/{\color{red} vendor} user}, I want to be able to \textbf{sign-up} using my \texttt{Facebook}/\texttt{Google} \textbf{account},\\
\indent
\textbf{so that} I don't have to \textbf{fill up forms} and also \textbf{sync my data} between different online platforms.
\end{userStory}

\begin{userStory}[Simultaneous Client/Vendor Account]
\textbf{As a {\color{green} client}/{\color{red} vendor} user}, I want to be able to consult the website as \textbf{both} a \textbf{purchasing client} and a \textbf{vendor} (in case I am both) on the website,\\
\indent
\textbf{so that} I get to \textbf{enjoy the website} in both \textbf{consumption} and \textbf{production} modes without having to sign-off and sign-in every time I want to switch between the modes.
\end{userStory}

\begin{userStory}[Footer Menu]
\textbf{As a {\color{green} client}/{\color{red} vendor} user}, I want to be able to consult a \textbf{menu in the footer of the index page} of the website,\\
\indent
\textbf{so that} I get to learn about the usage of the website through \textbf{FAQs}, understand what I can and cannot do through the \textbf{terms and conditions of usage}, get informed about the \textbf{creators of the website}, \textit{etc...}
\end{userStory}

\subsection{Home Page}
\begin{userStory}[Home Personal Settings]
\textbf{As a {\color{green} client}/{\color{red} vendor} user}, I want a \textbf{personalized user experience} with respect to my \textbf{preferences} and \textbf{history of purchases/sales} in the \textbf{settings},\\
\indent
\textbf{so that} I get to \textbf{visualize information} that are \textbf{relevant to my needs} while simultaneously preserving my \textbf{online privacy}.
\end{userStory}

\subsection{Searching}
\begin{userStory}[Search Results]
\textbf{As a {\color{green} client}/{\color{red} vendor} user}, I want to visualize information about products, vendors, and cells in my search results,\\
\indent
\textbf{so that} I get to have the \textbf{necessary amount of information} about them while surfing for \textbf{products} to \textbf{purchase} or \textbf{vendors/hives} to \textbf{consult}.
\end{userStory}

\begin{userStory}[Search Parameters]
\textbf{As a {\color{green} client}/{\color{red} vendor} user}, I want to \textbf{sort} my \textbf{search results} according to \textbf{parameters} such as stock information, harvest date, expiry date, price, proximity, popularity, vendor, category, list of similar products, etc...,\\
\indent
\textbf{so that} I get to \textbf{personalize} my \textbf{search results} according to my \textbf{needs}.
\end{userStory}

\begin{userStory}[Search Features]
\textbf{As a {\color{green} client}/{\color{red} vendor} user}, I want to use some \textbf{searching features} like \textbf{auto-completion}, \textbf{highlighting}, \textbf{visuals}, etc...,\\
\indent
\textbf{so that} I get to \textbf{search quickly, easily and intuitively} through for information.
\end{userStory}

\subsection{Communication}
\subsubsection{Instant Messaging}
\begin{userStory}[Client $\leftrightarrows$ Vendor and Vendor $\leftrightarrows$ Client/Vendor IM]
\textbf{As a {\color{green} client}/{\color{red} vendor} user}, I want to be able to communicate with a specific client/vendor privately in an instant message environment,\\
\indent
\textbf{so that} I get to \textbf{inquire more about specific information} concerning \textbf{products} or \textbf{hive cells} or other topics.
\end{userStory}

\subsubsection{Email}
\begin{userStory}[Client $\leftrightarrows$ Vendor and Vendor $\leftrightarrows$ Client/Vendor Email]
\textbf{As a {\color{green} client}/{\color{red} vendor} user}, I want to be able to communicate with a specific client/vendor privately via email,\\
\indent
\textbf{so that} my inquiries get to \textbf{reach them as soon as possible} on their \textbf{emails} in case they didn't consult their website account \textbf{regularly}.
\end{userStory}

\subsection{Products \& Logistics}
\subsubsection{Product Review}
\begin{userStory}[Product Reviews]
\textbf{As a {\color{green} client}/{\color{red} vendor} user}, I want to \textbf{consult reviews} (As a {\color{green} client}/{\color{red} vendor} user) about \textbf{products} and write them (As a {\color{green} client} user only),\\
\indent
\textbf{so that} I get to make \textbf{informed purchasing decisions} and \textbf{evaluate the experience} to benefit other future users.
\end{userStory}

\subsubsection{Product Management}
\begin{userStory}[Product Definition and Online Storage]
\textbf{As a {\color{red} vendor} user}, I want to \textbf{define} my \textbf{product selection} according to specific \textbf{descriptive properties} allowing me to divulge as much information about a product as possible,\\
\indent
\textbf{so that} I get to \textbf{maximize transparency} about my products and \textbf{gain customer loyalty} while \textbf{easily and intuitively managing}\footnote{creating, modifying, removing, adding} \textbf{my product selection} through the website.
\end{userStory}

\begin{userStory}[Basket Offers]
\textbf{As a {\color{red} vendor} user}, I would like to \textbf{propose baskets of differents products},\\
\indent
\textbf{so that} I get to offer a \textbf{diversified selection of my products} and \textbf{increase revenue}.
\end{userStory}

\begin{userStory}[Periodical Product Reports]
\textbf{As a {\color{red} vendor} user}, I want to have access to \textbf{statistical reports} about the \textbf{movements of products and stocks},\\
\indent
\textbf{so that} I get to \textbf{analyze the market} and \textbf{define} my \textbf{supply and demand methodology} accordingly.
\end{userStory}

\subsubsection{Logistics}
\begin{userStory}[Depletion of Stock Policy]
\textbf{As a {\color{red} vendor} user}, I want to \textbf{maximize my selling rate} to the point of stocks' \textbf{near-depletion},\\
\indent
\textbf{so that} I get to have the \textbf{least surplus of products in my stock as possible} and make more profit.
\end{userStory}

\begin{userStory}[Hive Product-Sharing Policy]
\textbf{As a {\color{red} vendor} user}, I want to have the possibility of \textbf{exchanging my products with vendors from other cells in the hive}, to offer their products in my cells and have my products offered in theirs,\\
\indent
\textbf{so that} we all benefit from a \textbf{mutual market expansion} and extended revenue surface.
\end{userStory}

\begin{userStory}[Relay and Location-Independent Delivery System]
\textbf{As a {\color{green} client} user}, I want to be capable of having my \textbf{purchases delivered} to a \textbf{desired location nearby a cell collection event} or to a \textbf{fixed relay center of distribution} if possible,\\
\indent
\textbf{so that} I get to collect my purchases \textbf{conveniently} wherever and whenever possible, without having to attend a hive cell collection event myself.
\end{userStory}

\begin{userStory}[Cell Collection Notifications]
\textbf{As a {\color{green} client}/{\color{red} vendor} user}, I want to have the option of \textbf{receiving notifications} about \textbf{hive cell collection events near me}, regardless whether or not I'm supposed to participate in them (not necessarily having purchased anything that I have to collect),\\
\indent
\textbf{so that} I get to know when and where to pick up my purchased products or simply be in touch with \textbf{nearby activity}.
\end{userStory}

\subsection{Order \& Payment}
\subsubsection{Order}
\begin{userStory}[Shopping Cart]
\textbf{As a {\color{green} client} user}, I would like to add the products I wish to purchase to a \textbf{virtual shopping cart},\\
\indent
\textbf{so that} I get to follow my \textbf{shopping progress} and visualize the \textbf{quantity of selected products}, their \textbf{individual prices}, and their \textbf{total price}.
\end{userStory}

\begin{userStory}[Time of Collection Selection]
\textbf{As a {\color{green} client} user}, I would like to choose \textbf{when to collect my purchased products} from the available time slots,\\
\indent
\textbf{so that} I get to collect them \textbf{conveniently} without troubling my personal schedule.
\end{userStory}

\begin{userStory}[Purchase Deadline]
\textbf{As a {\color{red} vendor} user}, I would like to impose specific \textbf{deadlines} on certain product orders,\\
\indent
\textbf{so that} I get to \textbf{customize} my \textbf{supply and demand parameters} while \textbf{processing pending orders} accordingly.
\end{userStory}

\begin{userStory}[Delayed Orders]
\textbf{As a {\color{red} vendor} user}, I would like to offer my customers the chance to make \textbf{delayed orders} for certain \textbf{out-of-stock products},\\
\indent
\textbf{so that} I don't lose my share of the market when certain stocks of products are depleted.
\end{userStory}

\begin{userStory}[Order Validation]
\textbf{As a {\color{red} vendor} user}, I would like to \textbf{manually} or \textbf{automatically validate purchase transactions},\\
\indent
\textbf{so that} I get to \textbf{customize} my \textbf{control} over the \textbf{transactions} according to my \textbf{products stocks}.
\end{userStory}

\subsubsection{Payment}
\begin{userStory}[Payment Methods]
\textbf{As a {\color{green} client} user}, I would like to have \textbf{secure online payment methods} through my \textbf{personal bank}/\texttt{PayPal} \textbf{account},\\
\indent
\textbf{so that} I \textbf{protect my financial credentials} and \textbf{complete my purchases reliably}.
\end{userStory}

\begin{userStory}[Digital Wallet]
\textbf{As a {\color{green} client} user}, I would like to have a \textbf{digital wallet associated to my own personal account} that contains digital currency points I collect from my website activity,\\
\indent
\textbf{so that} I benefit from \textbf{reductions} while purchasing certain products, defined according to the \textbf{number of points} I have \textbf{collected} through my website activity.
\end{userStory}

\begin{userStory}[Vendor Automatic Money Transfer]
\textbf{As a {\color{red} vendor} user}, I would like to have the \textbf{money} gained at the end of a transaction \textbf{transferred directly} into my bank/\texttt{Paypal} account,\\
\indent
\textbf{so that} I get to \textbf{update} my bank \textbf{balance automatically}.
\end{userStory}

\begin{userStory}[Receipt]
\textbf{As a {\color{green} client}/{\color{red} vendor} user}, I would like to receive a \textbf{receipt} at the end of a \textbf{transaction} via \textbf{sms} and \textbf{email}, and have my \textbf{history of purchases/sales updated},\\
\indent
\textbf{so that} I get to keep \textbf{track} of my \textbf{purchases/sales} through \textbf{different media} for larger \textbf{accessibility} and \textbf{data integrity}.
\end{userStory}

\begin{userStory}[Purchase Cancellation \& Reimbursment Policy]
\textbf{As a {\color{green} client} user}, I would like to have the possibility of \textbf{cancelling a purchase} within a \textbf{specific period} of its occurrence,\\
\indent
\textbf{so that} I would get a \textbf{full/partial reimbursement} following the faulty purchase.
\end{userStory}

%----------------------------------------------------------------------------------------
%   DIAGRAMMES USE-CASE
%----------------------------------------------------------------------------------------
\section{Diagrammes \textit{use-case}}
\subsection{Page d'accueil du site}
Le diagrammes \textit{use-case} correspondant aux \textit{user-stories} sur la page d'accueil du site est dans la figure \ref{fig:index_page_case_diagram}.
\begin{figure}[!ht]
  \centering
  \begin{tikzpicture}
    \umlactor{User}
    \umlactor[x=12, y=-3]{External Sign-up Platform}
    \umlactor[x=12, y=-1]{Database}
    \begin{umlsystem}[x=4, fill=red!10]{index page}
      \umlusecase{Sign-up}
      \umlusecase[y=-2, width=2.5cm]{Sign-up via external plaftorm}
      \umlusecase[y=-5]{Sign-in}
      \umlusecase[x=4, width=1.5cm]{Create database entry}
      \umlusecase[x=4, y=-4, width=1.5cm]{Verify credentials}
      \umlusecase[y=-7]{Footer menu}
    \end{umlsystem}
    \umlassoc{User}{usecase-1}
    \umlassoc{User}{usecase-2}
    \umlassoc{User}{usecase-3}
    \umlassoc{User}{usecase-6}
    \umlassoc{External Sign-up Platform}{usecase-2}
    \umlassoc{Database}{usecase-4}
    \umlassoc{Database}{usecase-5}
    \umlinclude{usecase-1}{usecase-4}
    \umlinclude{usecase-2}{usecase-4}
    \umlinclude{usecase-3}{usecase-5}
  \end{tikzpicture}
  \caption{Use case diagram of index page functionalities.}
  \label{fig:index_page_case_diagram}
\end{figure}

\subsection{Page d'accueil des utilisateurs}
Le diagramme \textit{use-case} correspondant aux \textit{user-stories} sur la page d'accueil des utilisateurs (client/fournisseur) du site est dans la figure \ref{fig:home_page_use_case_diagram}
\begin{figure}[!ht]
  \centering
  \begin{tikzpicture}
    \setcounter{tikzumlUseCaseNum}{0}
    \umlactor{User}
    \umlactor[y=-4]{Vendor}
    \umlactor[x=9, y=-1]{Database}
    \begin{umlsystem}[x=4, fill=green!10]{home page}
      \umlusecase[width=2cm]{Display Products, Hive Events}
      \umlusecase[x=2, y=-3, width=2cm]{Display Stock/Hive Information}
      \umlusecase[y=-6]{Search}
      \umlusecase[y=-8]{Footer menu}
    \end{umlsystem}
    \umlinherit{Vendor}{User}
    \umlextend{usecase-2}{usecase-1}
    \umlassoc{User}{usecase-1}
    \umlassoc{Vendor}{usecase-2}
    \umlassoc{User}{usecase-3}
    \umlassoc{User}{usecase-4}
    \umlassoc{Database}{usecase-1}
    \umlassoc{Database}{usecase-3}
  \end{tikzpicture}
  \caption{Use case diagram of Home Page functionalities.}
  \label{fig:home_page_use_case_diagram}
\end{figure}

\subsection{Recherche}
Le diagramme \textit{use-case} correspondant aux \textit{user-stories} sur la recherche au sein du site est dans la figure \ref{fig:search_use_case_diagram}
\begin{figure}[!ht]
  \centering
  \begin{tikzpicture}
    \setcounter{tikzumlUseCaseNum}{0}
    \umlactor[x=-1]{User}
    \umlactor[x=10]{Database}
    \begin{umlsystem}[x=4, fill=green!10]{home page}
      \umlusecase{Search}
      \umlusecase[x=-2, y=-3, width=2cm]{Choose Search Parameters}
      \umlusecase[x=2, y=-3, width=2cm]{Activate Search Features}
    \end{umlsystem}
    \umlextend[name=ext]{usecase-2}{usecase-1}
    \umlextend[name=ext2]{usecase-3}{usecase-1}
    \umlassoc{User}{usecase-1}
    \umlassoc{User}{usecase-2}
    \umlassoc{User}{usecase-3}
    \umlassoc{Database}{usecase-1}
    \umlassoc{Database}{usecase-2}
    \umlassoc{Database}{usecase-3}
    \umlnote[x=-2, y=-6]{ext-1}{default parameters, else check what parameters to use}
    \umlnote[x=10, y=-6]{ext2-1}{activated by default, else disable in settings}
  \end{tikzpicture}
  \caption{Use case diagram of searching functionality.}
  \label{fig:search_use_case_diagram}
\end{figure}

\subsection{Communication}
Les diagrammes \textit{use-case} correspondant aux \textit{user-stories} sur la communication au sein du site sont respectivement dans les figures \ref{fig:instant_messaging_use_case_diagram} et \ref{fig:email_use_case_diagram}.
\begin{figure}[!ht]
  \centering
  \begin{tikzpicture}
    \setcounter{tikzumlUseCaseNum}{0}
    \umlactor{User}
    \umlactor[x=8]{Vendor}
    \begin{umlsystem}[x=4, fill=green!10]{home page}
      \umlusecase{Chat privately}
    \end{umlsystem}
    \umlassoc{User}{usecase-1}
    \umlassoc{Vendor}{usecase-1}
  \end{tikzpicture}
  \caption{Use case diagram of private instant messaging between client/vendor and vendor.}
  \label{fig:instant_messaging_use_case_diagram}
\end{figure}

\begin{figure}[!ht]
  \centering
  \begin{tikzpicture}
    \setcounter{tikzumlUseCaseNum}{0}
    \umlactor{User}
    \umlactor[x=7]{Vendor}
    \umlactor[x=4, y=3]{Email Service Interface}
    \umlactor[x=14]{Email Database}
    \begin{umlsystem}[x=4, fill=green!10]{home page}
      \umlusecase{Send email}
    \end{umlsystem}
    \begin{umlsystem}[x=10, fill=blue!10]{email system}
      \umlusecase{Receive emails}
      \umlusecase[y=-2]{Check inbox}
    \end{umlsystem}
    \umlassoc{User}{usecase-1}
    \umlassoc{Email Service Interface}{usecase-1}
    \umlassoc{Email Service Interface}{usecase-2}
    \umlassoc{Vendor}{usecase-3}
    \umlassoc{Email Database}{usecase-2}
    \umlassoc{Email Database}{usecase-3}
  \end{tikzpicture}
  \caption{Use case diagram of an email functionality between client/vendor and vendor.}
  \label{fig:email_use_case_diagram}
\end{figure}

\subsection{Produits}
Les diagrammes \textit{use-case} correspondant aux \textit{user-stories} sur les produits sont respectivement dans les figures \ref{fig:product_review_use_case_diagram} et \ref{fig:product_management_use_case_diagram}.
\begin{figure}[!ht]
  \centering
  \begin{tikzpicture}
    \setcounter{tikzumlUseCaseNum}{0}
    \umlactor{User}
    \umlactor[y=-4]{Client}
    \umlactor[x=8, y=-1]{Database}
    \begin{umlsystem}[x=4, fill=green!10]{home page}
      \umlusecase{Search}
      \umlusecase[y=-3, width=2cm]{Search for Products}
      \umlusecase[y=-6]{Review Products}
    \end{umlsystem}
    \umlinherit{Client}{User}
    \umlextend{usecase-2}{usecase-1}
    \umlextend[name=ext]{usecase-3}{usecase-2}
    \umlassoc{User}{usecase-1}
    \umlassoc{Client}{usecase-3}
    \umlassoc{Database}{usecase-1}
    \umlassoc{Database}{usecase-3}
    \umlnote[x=9, y=-6]{ext-1}{Only if the user is a client having already purchased the product}
  \end{tikzpicture}
  \caption{Use case diagram of product reviewing functionality.}
  \label{fig:product_review_use_case_diagram}
\end{figure}

\begin{figure}[!ht]
  \centering
  \begin{tikzpicture}
    \setcounter{tikzumlUseCaseNum}{0}
    \umlactor{Vendor}
    \umlactor[x=8, y=-1]{Database}
    \begin{umlsystem}[x=4, fill=green!10]{home page}
      \umlusecase[width=3cm]{Add/Delete/Modify Products}
      \umlusecase[y=-3, width=3cm]{Add/Delete/Modify Baskets}
      \umlusecase[y=-6]{Check Stocks}
      \umlusecase[y=-9]{Check Reports}
    \end{umlsystem}
    \umlextend[name=ext]{usecase-2}{usecase-1}
    \umlextend{usecase-4}{usecase-3}
    \umlassoc{User}{usecase-1}
    \umlassoc{User}{usecase-2}
    \umlassoc{User}{usecase-3}
    \umlassoc{User}{usecase-4}
    \umlassoc{Database}{usecase-1}
    \umlassoc{Database}{usecase-2}
    \umlassoc{Database}{usecase-3}
    \umlassoc{Database}{usecase-4}
    \umlnote[x=-2, y=-4]{ext-1}{Products have to exist in the database prior to the introduction of baskets}
  \end{tikzpicture}
  \caption{Use case diagram of product management functionality.}
  \label{fig:product_management_use_case_diagram}
\end{figure}

\subsection{Commandes \& Paiement}
Les diagrammes \textit{use-case} correspondant aux \textit{user-stories} sur les commandes et paiements sont respectivement dans les figures \ref{fig:order_payment_client_use_case_diagram} et \ref{fig:order_payment_vendor_use_case_diagram}.
\begin{figure}[!ht]
  \centering
  \begin{tikzpicture}
    \setcounter{tikzumlUseCaseNum}{0}
    \umlactor{Client}
    \umlactor[y=-4]{Database}
    \umlactor[x=14, y=-6.2]{Vendor User Database}
    \umlactor[x=14, y=-11]{Money Payment Platform}
    \umlactor[x=8, y=-14]{Bill Reception Platform}
    \umlactor[x=4.5, y=-14]{Bank/PayPal}
    \begin{umlsystem}[x=4, fill=green!10]{client home page}
      \umlusecase[width=2cm] {Add Items to Shopping Cart} %1
      \umlusecase[x=5, width=2cm]{Delete Items from Shopping Cart} %2
      \umlusecase[x=3.5, y=-3, width=2cm]{Choose Time of Collection} %3
      \umlusecase[x=5, y=-6, width=3cm]{Choose Payment Method} %4
      \umlusecase[x=5, y=-9, width=3cm]{Get Reductions from Wallet Points} %5
      \umlusecase[y=-12]{Check Wallet} %6
      \umlusecase[y=-9, width=3cm]{Checkout from Payment} %7
      \umlusecase[y=-6, width=3cm]{Update Purchases' History} %8
      \umlusecase[x=5, y=-12, width=3cm]{Receive Bill} %9
    \end{umlsystem}
    \begin{umlsystem}[x=14.2, fill=red!10]{vendor home page}
      \umlusecase[width=1.5cm]{Send a Bill} %10
    \end{umlsystem}
    \umlextend{usecase-2}{usecase-1}
    \umlextend{usecase-5}{usecase-4}
    \umlinclude{usecase-1}{usecase-3}
    \umlinclude{usecase-3}{usecase-4}
    \umlinclude{usecase-7}{usecase-8}
    \umlinclude[pos stereo=0.2]{usecase-9}{usecase-8}
    \umlassoc{Client}{usecase-1}
    \umlassoc{Client}{usecase-6}
    \umlassoc{Money Payment Platform}{usecase-4}
    \umlHVassoc{Money Payment Platform}{usecase-7}
    \umlassoc{Vendor User Database}{usecase-7}
    \umlassoc{Vendor User Database}{usecase-9}
    \umlassoc[name=send]{Vendor User Database}{usecase-10}
    \umlassoc{Bill Reception Platform}{usecase-9}
    \umlassoc{Bank/PayPal}{usecase-9}
    \umlassoc{Database}{usecase-8}
    \umlnote[x=12, y=-3]{send-1}{only if transaction validated by vendor}
  \end{tikzpicture}
  \caption{Use case diagram of Order and Payment functionalities viewed from the client user side.}
  \label{fig:order_payment_client_use_case_diagram}
\end{figure}

\begin{figure}[!ht]
  \centering
  \begin{tikzpicture}
    \setcounter{tikzumlUseCaseNum}{0}
    \umlactor{Vendor}
    \umlactor[x=13, y=-2]{Database}
    \umlactor[y=-13]{Client User Database}
    \umlactor[x=13, y=-11]{Bill Reception Platform}
    \umlactor[x=4.5, y=-13]{Bank/PayPal}
    \begin{umlsystem}[x=4, fill=red!10]{vendor home page}
      \umlusecase[width=2cm]{Check Stocks}
      \umlusecase[x=5, width=2cm]{Allow Delayed Orders}
      \umlusecase[x=3.5, y=-3, width=3cm]{Set Purchase Deadline}
      \umlusecase[x=4, y=-6, width=3cm]{Check Pending Orders}
      \umlusecase[y=-8, width=2cm]{Process Pending Orders}
      \umlusecase[x=5, y=-8, width=2cm]{Send Bill}
      \umlusecase[x=4.5, y=-11, width=2cm]{Update Sales' History}
    \end{umlsystem}
    \umlextend{usecase-2}{usecase-1}
    \umlextend[name=deadline]{usecase-3}{usecase-4}
    \umlextend[name=checkPending1]{usecase-4}{usecase-5}
    \umlinclude{usecase-5}{usecase-6}
    \umlinclude{usecase-6}{usecase-7}
    \umlassoc{Vendor}{usecase-1}
    \umlassoc[name=checkPending2]{Vendor}{usecase-4}
    \umlassoc[name=process]{Vendor}{usecase-5}
    \umlassoc{Database}{usecase-2}
    \umlassoc{Database}{usecase-3}
    \umlassoc{Database}{usecase-4}
    \umlassoc{Database}{usecase-7}
    \umlassoc{Bill Reception Platform}{usecase-6}
    \umlassoc{Bank/PayPal}{usecase-6}
    \umlassoc{Client User Database}{usecase-6}
    \umlnote[x=-1, y=-5]{checkPending1-1}{only if validating transactions is set to manual}
    \umlnote[x=-1, y=-5]{checkPending2-1}{only if validating transactions is set to manual}
    \umlnote[x=-1, y=-8]{process-1}{only if validating transactions is set to automatic}
    \umlnote[x=-1, y=-11]{deadline-1}{according to the number of pending orders}
  \end{tikzpicture}
  \caption{Use case diagram of Order and Payment functionalities viewed from the vendor user side.}
  \label{fig:order_payment_vendor_use_case_diagram}
\end{figure}
%----------------------------------------------------------------------------------------
%   SD PROPOSÉE (CELLULE ET RUCHE)
%----------------------------------------------------------------------------------------
\section{Structure de données proposée (\texttt{Cellule} et \texttt{Ruche})}
  %BACHAR

%----------------------------------------------------------------------------------------
%   MODÈLE EA
%----------------------------------------------------------------------------------------
\section{Modèle EA}
  %OTHMANE

%----------------------------------------------------------------------------------------
%   SCHÉMA DE BD
%----------------------------------------------------------------------------------------
\section{Schéma de base de données}
  %WISSEM

%----------------------------------------------------------------------------------------
%   STORYBOARD
%----------------------------------------------------------------------------------------
\section{Storyboard}
  %WISSEM

%----------------------------------------------------------------------------------------
%   CONCLUSION
%----------------------------------------------------------------------------------------
\chapter{Conclusion}
%----------------------------------------------------------------------------------------
%   IMPLÉMENTATION PRÉVUE
%----------------------------------------------------------------------------------------
\section{Implémentation prévue}
  %BACHAR

%----------------------------------------------------------------------------------------
%   PERSPECTIVES
%----------------------------------------------------------------------------------------
\section{Perspectives}
  %BACHAR

%----------------------------------------------------------------------------------------
%   BIBLIOGRAPHIE
%----------------------------------------------------------------------------------------
\bibliographystyle{agsm}
\bibliography{mybib}

\end{document}
